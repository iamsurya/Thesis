\chapter{C Code used to program the micrcontroller.}
\label{Chap:CCODE}
\lstset{
  language=C,
  aboveskip=3mm,
  belowskip=3mm,
  showstringspaces=false,
  columns=flexible,
  basicstyle={\small\ttfamily},
  numbers=none,
  numberstyle=\tiny\color{black},
  keywordstyle=\color{black},
  commentstyle=\color{black},
  stringstyle=\color{black},
  breaklines=false,
  breakatwhitespace=true,
  tabsize=3,
frame=b
}
\begin{lstlisting}[caption=main.h,label=Code1]
/* Define functions */
#include <msp430f248.h>
#include "mpu6000.h"
#include "dataflash.h"

/* General Defines */
#define         LED             BIT0
#define         BTN             BIT4    /* Button on P1.4 */

/* SPI Pin defines */
#define         nSS             BIT0    /* Enable Pin for Sensor */
#define         SPI_SIMO        BIT1
#define         SPI_SOMI        BIT2
#define         SPI_CLK         BIT3
#define         mSS             BIT4    /* Enable Pin for Memory Chip */

/* SPI commands / vars */
#define         MPU_READ        0x80
/* PageSize for current Memory is set to 1024, but 6 * 170 = 1020*/
/* so we use this for all counters */
#define PAGESIZE 1020 				
/* UART Pins and vars */
#define RXPIN   BIT5
#define TXPIN   BIT4
#define RS232_ESC       27
#define ASCII0  0x30

/* Variables for Button Debouncing */
/* 1 Not waiting for Debounce Timer */
/* 0 Button was pressed very soon */
unsigned char debounce;               

/* Variables for SPI */
unsigned char read = 0;
unsigned char PageAddress_H = 0;
unsigned char PageAddress_L = 0;

/* Variables for Main Program and to store Data */
/* ActionModes: 0 = LPM3, 1 = Switch between Blink / No Blink modes*/
/* 2 = Switch Blink Frequencies, 3 = SPI reading*/
unsigned char ActionMode = 0;
/* Stores data to output to Memory.*/
/* Size of Buffer and Block on Memory is PAGESIZE bytes by default */
signed char SensorData[PAGESIZE];
/* The current page number we are reading or writing */
unsigned int CurrentPage = 0;
/* Variable used for counters.*/	
unsigned int ctr = 0;		
/* 1 = Writing Sensor Data to Memory, 0 = Not Writing Data */
unsigned char WritingMode = 0;
/* 1 = Writing Sensor Data to Memory, 0 = Not Writing Data */
unsigned char SwitchOn = 0; 
unsigned char FORCESTOP = 0; 
void ONDANCE();
void OFFDANCE();

/* Timer for reading sensors */
 /* Blinking frequency / timer on startup 0x1000 is 1 second */
unsigned int BaseTime = 2184; 	  
/* Flag is set if Timer interrupts an SPI operation */
unsigned char ReadingSensor = 0x00; 

/* Variables to store Timer Counts to evaluate communication times */
unsigned int StartTime = 0;
unsigned int EndTime = 0;
unsigned int Time = 0;

/* Variables for logging to UART */
unsigned long ReadIndex = 0;
char PhoneViewStart[] = "START 2014-21-04   07:30:20";
char PhoneViewEnd[] = "END 2014-21-04   20:05:05";
unsigned char UART_Working = 0;
unsigned char UART_data[2];

/* Functions */

/* Sensor */
unsigned char Sensor_TXRX(unsigned char add, unsigned char val);
unsigned char _Sensor_write(unsigned char add, unsigned char val);
unsigned char _Sensor_read(unsigned char add);

/* General */
void JustDance();
void ReadPageNumberFromFlash();
void WritePageNumberToFlash();

/* Memory */
unsigned char MEM_TXRX(unsigned char data);
unsigned char Mem_ReadID();
void FlushMemory(void);
void Mem_WriteToBuffer();
void Mem_ReadFromBuffer();
void Mem_ReadFromMem();
void Mem_BufferToPage();

void Mem_ReadAllBinary();

/* UART Functions */

void ClearScreen();
void UART_SendValue(signed char);
void UART_SendChar(unsigned char);
void UART_SendIndex(unsigned long num);
void UART_SendTime(unsigned long index);
void UART_SendValue2(unsigned char num);

void UART_SendValue2(unsigned char num)
{
  unsigned char p = 0;
  p = (unsigned char) num / 10;
  UART_SendChar( p + ASCII0 );
  num = num - (p * 10);
  p = (unsigned char) num;
  UART_SendChar( p + ASCII0 );  
}
\end{lstlisting}
\pagebreak
\begin{lstlisting}[caption=dataflash.h,label=Code2]
/* Dataflash Commands. Source : http://playground.arduino.cc/Code/Dataflash */

#define FlashPageRead		0x52	// Main memory page read
#define FlashToBuf1Transfer   	0x53	// Main memory page to buffer 1 transfer
#define Buf1Read             		0xD1	// Buffer 1 read
#define FlashToBuf2Transfer 	0x55	// Main memory page to buffer 2 transfer
#define Buf2Read			0x56	// Buffer 2 read
#define StatusReg			0x57	// Status register
#define AutoPageReWrBuf1	0x58	// Auto page rewrite through buffer 1
#define AutoPageReWrBuf2	0x59	// Auto page rewrite through buffer 2
#define FlashToBuf1Compare      0x60	// Main memory page to buffer 1 compare
#define FlashToBuf2Compare	0x61	// Main memory page to buffer 2 compare
#define ContArrayRead		0x68	// Continuous Array Read (Note : Only A/B-parts supported)
#define FlashProgBuf1		0x82	// Main memory page program through buffer 1
#define Buf1ToFlashWE   		0x83	// Buffer 1 to main memory page program with built-in erase
#define Buf1Write			0x84	// Buffer 1 write
#define FlashProgBuf2		0x85	// Main memory page program through buffer 2
#define Buf2ToFlashWE   		0x86	// Buffer 2 to main memory page program with built-in erase
#define Buf2Write			0x87	// Buffer 2 write
#define Buf1ToFlash     		0x88	// Buffer 1 to main memory page program w/o built-in erase
#define Buf2ToFlash		0x89	// Buffer 2 to main memory page program w/o built-in erase
#define PageErase             	0x81	// Page erase, added by Martin Thomas

\end{lstlisting}
\pagebreak
\begin{lstlisting}[caption=MPU6000 Defines,label=Code3]
/* Code from https://code.google.com/p/gentlenav/ */
#ifndef __MPU6000_H__
#define __MPU6000_H__


// MPU6000 registers
#define MPUREG_AUX_VDDIO  0x01
#define MPUREG_XG_OFFS_TC 0x00
#define MPUREG_YG_OFFS_TC 0x01
#define MPUREG_ZG_OFFS_TC 0x02
#define MPUREG_X_FINE_GAIN 0x03
#define MPUREG_Y_FINE_GAIN 0x04
#define MPUREG_Z_FINE_GAIN 0x05
#define MPUREG_XA_OFFS_H 0x06
#define MPUREG_XA_OFFS_L 0x07
#define MPUREG_YA_OFFS_H 0x08
#define MPUREG_YA_OFFS_L 0x09
#define MPUREG_ZA_OFFS_H 0x0A
#define MPUREG_ZA_OFFS_L 0x0B
#define MPUREG_PRODUCT_ID 0x0C
#define MPUREG_XG_OFFS_USRH 0x13
#define MPUREG_XG_OFFS_USRL 0x14
#define MPUREG_YG_OFFS_USRH 0x15
#define MPUREG_YG_OFFS_USRL 0x16
#define MPUREG_ZG_OFFS_USRH 0x17
#define MPUREG_ZG_OFFS_USRL 0x18
#define MPUREG_SMPLRT_DIV 0x19
#define MPUREG_CONFIG 0x1A
#define MPUREG_GYRO_CONFIG 0x1B
#define MPUREG_ACCEL_CONFIG 0x1C
#define MPUREG_INT_PIN_CFG 0x37
#define MPUREG_INT_ENABLE 0x38
#define MPUREG_ACCEL_XOUT_H 0x3B
#define MPUREG_ACCEL_XOUT_L 0x3C
#define MPUREG_ACCEL_YOUT_H 0x3D
#define MPUREG_ACCEL_YOUT_L 0x3E
#define MPUREG_ACCEL_ZOUT_H 0x3F
#define MPUREG_ACCEL_ZOUT_L 0x40
#define MPUREG_TEMP_OUT_H 0x41
#define MPUREG_TEMP_OUT_L 0x42
#define MPUREG_GYRO_XOUT_H 0x43
#define MPUREG_GYRO_XOUT_L 0x44
#define MPUREG_GYRO_YOUT_H 0x45
#define MPUREG_GYRO_YOUT_L 0x46
#define MPUREG_GYRO_ZOUT_H 0x47
#define MPUREG_GYRO_ZOUT_L 0x48
#define MPUREG_USER_CTRL 0x6A
#define MPUREG_PWR_MGMT_1 0x6B
#define MPUREG_PWR_MGMT_2 0x6C
#define MPUREG_BANK_SEL 0x6D
#define MPUREG_MEM_START_ADDR 0x6E
#define MPUREG_MEM_R_W 0x6F
#define MPUREG_DMP_CFG_1 0x70
#define MPUREG_DMP_CFG_2 0x71
#define MPUREG_FIFO_COUNTH 0x72
#define MPUREG_FIFO_COUNTL 0x73
#define MPUREG_FIFO_R_W 0x74
#define MPUREG_WHOAMI 0x75

// Configuration bits MPU6000
#define BIT_SLEEP                                       0x40
#define BIT_H_RESET                             0x80
#define BITS_CLKSEL                             0x07
#define MPU_CLK_SEL_PLLGYROX            0x01
#define MPU_CLK_SEL_PLLGYROZ            0x03
#define MPU_EXT_SYNC_GYROX                      0x02
#define BITS_FS_250DPS              0x00
#define BITS_FS_500DPS              0x08
#define BITS_FS_1000DPS             0x10
#define BITS_FS_2000DPS             0x18
#define BITS_FS_2G                  0x00
#define BITS_FS_4G                  0x08
#define BITS_FS_8G                  0x10
#define BITS_FS_16G                 0x18
#define BITS_FS_MASK                0x18
#define BITS_DLPF_CFG_256HZ_NOLPF2  0x00
#define BITS_DLPF_CFG_188HZ         0x01
#define BITS_DLPF_CFG_98HZ          0x02
#define BITS_DLPF_CFG_42HZ          0x03
#define BITS_DLPF_CFG_20HZ          0x04
#define BITS_DLPF_CFG_10HZ          0x05
#define BITS_DLPF_CFG_5HZ           0x06
#define BITS_DLPF_CFG_2100HZ_NOLPF  0x07
#define BITS_DLPF_CFG_MASK          0x07
//#define BIT_INT_ANYRD_2CLEAR        0x10
//#define BIT_RAW_RDY_EN              0x01
#define BIT_I2C_IF_DIS              0x10

// Register 55 - INT Pin / Bypass Enable Configuration (INT_PIN_CFG)
#define BIT_INT_LEVEL        0x80
#define BIT_INT_OPEN         0x40
#define BIT_LATCH_INT_EN     0x20
#define BIT_INT_RD_CLEAR     0x10
#define BIT_FSYNC_INT_LEVEL  0x08
#define BIT_FSYNC_INT_EN     0x04
#define BIT_I2C_BYPASS_EN    0x02
#define BIT_CLKOUT_EN        0x01

// Register 56 - Interrupt Enable (INT_ENABLE)
#define BIT_FF_EN            0x80
#define BIT_MOT_EN           0x40
#define BIT_ZMOT_EN          0x20
#define BIT_FIFO_OFLOW_EN    0x10
#define BIT_I2C_MST_INT_EN   0x08
#define BIT_DATA_RDY_EN      0x01

// Register 58 - Interrupt Status (INT_STATUS)
#define BIT_FF_INT           0x80
#define BIT_MOT_INT          0x40
#define BIT_ZMOT_INT         0x20
#define BIT_FIFO_OFLOW_INT   0x10
#define BIT_I2C_MST_INT      0x08
#define BIT_DATA_RDY_INT     0x01


// DMP output rate constants
#define MPU6000_200HZ 0    // default value
#define MPU6000_100HZ 1
#define MPU6000_66HZ 2
#define MPU6000_50HZ 3

#endif // __MPU6000_H__

\end{lstlisting}
\pagebreak
\begin{lstlisting}[caption=Main.c,label=Code4]
#include "main.h"
        
#define TIMETOWRITE 60
#define TIMETOREAD 60

#define PAGESTOWRITE 8189 // (TIMETOWRITE * 5.5)
#define PAGESTOREAD 8189 //(TIMETOREAD * 5.5)


/*
** Main Function. We setup the clocks
** and SPI, then leave the device into 
** sleep mode to be activated by the Timer Interrupt
*/
void main(void)
{
  
  /* Turn off Watchdog Timer */
  WDTCTL = WDTPW+WDTHOLD;                   

  BCSCTL3 |= XCAP_3;          /* Set the capacitance of the external crystal to 12.4pf. */  
  
   /* Initialize DCO */
  BCSCTL1 = CALBC1_1MHZ;                /* Set DCO to 1MHz */
  DCOCTL =  CALDCO_1MHZ;                /* Set DCO to 1MHz */
  __delay_cycles(200000);
  debounce = 1;
  // MCLK/3 for Flash Timing Generator (0x0040u) 
  /* Flash clock select: 1 - MCLK Divided by 2. */
  FCTL2 = FWKEY + FSSEL0 + FN1;             
  SVSCTL = 0x80;                       /* SVS to set at 2.8V */
  
  /* Initialize Port 1 For LED and Button */
  P1DIR |= LED;                 /* Set LED as output */
  P1REN |= BTN;                 /* Enable Pullup / Pulldown on BTN pin */
  P1OUT |= BTN;           /* Turn LED OFF Till the BTN is pressed, Pull up BTN pin */
  P1OUT &= ~LED;            
  P1IFG = 0;                    /* Clear any interrupts on P1 */
  P1IE  |= BTN;                 /* Set BTN as intterupt (default P1IES sets low to high )*/
  P1IES |= BTN;                 /* Sensitive to (H->L) */
  
  /* Enable SPI */
  P5SEL |= (BIT1 + BIT2 + BIT3);        /* Peripheral function instead of I/O */
  /* SPI Polarity = 1, MSB First, Master*/
  /* 3 Pin Mode, Synchronous Comm. UCCKPH = 0 is ~SPIPhase*/
  UCB1CTL0 = UCCKPL | UCMSB | UCMST | UCMODE_0 | UCSYNC; 
  UCB1CTL1 = UCSSEL_2 | UCSWRST;         /* Clock from SMCLK; hold SPI in reset */
  
  UCB1BR1 = 0x0;                          /* Upper byte of divider word */
  UCB1BR0 = 0x1;                         /* Clock = SMCLK / 10 = 100 KHz */
  
  UCB1CTL1 &= ~UCSWRST;                 /* Remove SPI reset to enable it*/
  
  /* Configure Pins for SPI. These commands might not be needed */
  P5DIR |= nSS | mSS | SPI_SOMI | SPI_CLK;    /* Output on the pins */      
  P5OUT |= nSS | mSS | SPI_SOMI | SPI_CLK;    /* Set pins to high */
  __delay_cycles(0x80);
  
  /* Read the Memory Device ID. Should be 1F */
  read = Mem_ReadID();

//  Mem_ReadAllBinary();
  /* Initialize UART */
  P3DIR = TXPIN;
  P3OUT = TXPIN;
  P3SEL = TXPIN | RXPIN;

  /* Values from MSP430x24x Demo - USCI_A0, 115200 UART Echo ISR, DCO SMCLK */
  
  UCA0CTL1 |= UCSSEL_2;                     // CLK = SMCLK
  UCA0BR0 = 8;                           // 32kHz/9600 = 3.41
  UCA0BR1 = 0x00;
  UCA0MCTL = UCBRS2 + UCBRS0;               // Modulation UCBRSx = 3
  UCA0CTL1 &= ~UCSWRST;                     // **Initialize USCI state machine**
  IE2 |= UCA0RXIE;                          // Enable USCI_A0 RX interrupt
  /* End UART init */

  /* Disable I2C on the sensor */
  _Sensor_write(MPUREG_USER_CTRL, BIT_I2C_IF_DIS);
  
  /* Check the Sensor Device ID */
  read = _Sensor_read(MPUREG_WHOAMI);

  /* Trap uC if Sensor doesn't ID correctly */
  /* Check WHOAMI Hopefully it is 0x68. Otherwise freak out. */
  if(read != 0x68) JustDance();                    

 
  /* Sensor might be sleeping, read Register with Sleep Bit */
  /* reset the bit and write it back */
  read = _Sensor_read(MPUREG_PWR_MGMT_1);
  /* Reset the Sleep Bit to wake it up */
  _Sensor_write(MPUREG_PWR_MGMT_1, (read & ~BIT_SLEEP));
  
  /* Erase all data SensorData */
  for(ctr = 0; ctr < PAGESIZE; ctr++)
    SensorData[ctr] = 0;

  /* Reset Counter to 0 after previous for loop */
  ctr = 0;
  
  /* 
  ** TimerA for Blinking, Button Debouncing, Long Press wait 
  ** TACTL = Timer A Control
  ** TASSEL_1 => Set Timer source to ACLK (ACLK is from LFXT) | MC_2 = MODE => Continuous
  ** TACLR => Clear Timer | ID_3 Divide Clock by 8 = > 4096 = 0x1000
  ** This does not start the timer (Set mode to not 0, and have not zero value in TACCR0)).
  */
  TACTL = TACLR;
  __delay_cycles(8254); /* Delay by 0.2 seconds DELAY CYCLES USES DCO @ 1MHZ*/
  TACTL = TASSEL_1 | MC_2; /* Run Timer on ACLK */

    /* Other Housekeepigng */
  __delay_cycles(8234);                 /* Delay 0.2s at 1Mhz to let clock settle */
  __enable_interrupt();

  OFFDANCE();

  /** Main Program **/
  
  while(1)
  {
    __low_power_mode_3();
    
    if(ActionMode == 1)                 /* Long Buttton Press from old Timer Code */
    {
      if(((WritingMode == 1) && ((P1IN & BTN) == 0)) || (FORCESTOP == 1))
      {
        /* Disable Sensor Reading */
        /* Set timer to disable */
        FORCESTOP = 0;
        TACCR0 = 0;
        TACCTL0 &= ~CCIE;
        OFFDANCE();
        WritingMode = 0;
        SwitchOn = 0;
        __enable_interrupt();
        __delay_cycles(600);
      }
      else if((WritingMode == 0) && ((P1IN & BTN) == 0))
      {
       /* Reset Memory count since device is off and LongPress */
        CurrentPage = 0;               
        WritePageNumberToFlash();
        ctr = 0;
        JustDance();
        SwitchOn = 0;
        
      }
      else if (SwitchOn == 1)
      {
        TACCR0 += (BaseTime);
        TACCTL0 |= CCIE;
        ONDANCE();
        
        P1OUT &= ~LED;
        WritingMode = 1;
        SwitchOn = 0;
      }
      ActionMode = 0;  
    }
    
    if(ActionMode == 2)                 /* Short Button Press from old Timer Code */
    {
      if(WritingMode == 0)              /* Device is off and shortpress */
      {
      SwitchOn = 1;
      }
      ActionMode = 0;
    }    
    
    if(ActionMode == 3)                  /* Timer A Freq at 15hz */
    {
      ReadingSensor = 1;
      /* 2nd to Last data should be -128, Last Should be 0 */
      if(CurrentPage == PAGESTOWRITE && ctr > 1007) 
      {
      SensorData[ctr++] = -128;
      SensorData[ctr++] = -128;
      SensorData[ctr++] = -128;
      SensorData[ctr++] = -128;
      SensorData[ctr++] = -128;
      SensorData[ctr++] = -128;
      JustDance();
      }
      else
      {
      /* Read X Acc */
      SensorData[ctr++] = _Sensor_read(MPUREG_ACCEL_XOUT_H); // rand--;//
      /* Read Y Acc */
      SensorData[ctr++] =  _Sensor_read(MPUREG_ACCEL_YOUT_H);
      /* Read Z Acc */
      SensorData[ctr++] =  _Sensor_read(MPUREG_ACCEL_ZOUT_H);
      /* Read X Gyro */
      SensorData[ctr++] =  _Sensor_read(MPUREG_GYRO_XOUT_H);
      /* Read Y Gyro */
      SensorData[ctr++] = _Sensor_read(MPUREG_GYRO_YOUT_H);
      /* Read Z Gyro */
      SensorData[ctr++] = _Sensor_read(MPUREG_GYRO_ZOUT_H);
      }
      

          if(ctr >= PAGESIZE)   /* The Buffer is full */
          {
	/* Disable interrupts b/c the timer might interrupt otherwise */
            __disable_interrupt();	
            P1OUT |= LED;
            ReadPageNumberFromFlash();
            StartTime = TAR;
            /* Write the stored SensorData to the Memory chip Buffer */
            Mem_WriteToBuffer();	
            /* Write the Buffer data to Page. Page Number is stored in Global CurrentPage */
            Mem_BufferToPage();		
            CurrentPage++;			/* Increment current page for next write */
            WritePageNumberToFlash();
            ctr=0;					/* Reset ctr so its starts at 0 for next page */
            EndTime = TAR;
            Time = EndTime - StartTime;
            /* Erase SensorData. Using new ctr variable b/c ctr is used in loop above */
            for(int actr = 0; actr < PAGESIZE; actr++)	
                SensorData[actr] = 0;
            ReadingSensor = 0; /* Flag set to 0 b/c we're done with communication */
            ActionMode = 0;
            P1OUT &= ~LED;
            /* Enable Interrupts for the timer to start logging data again */
            __enable_interrupt();	

          }
      
      ReadingSensor = 0; /* Flag set to 0 b/c we're done with communication */
      ActionMode = 0;
    }
  }
  
}

/*********************/
/* Code for MPU 6000 */
/*********************/

unsigned char _Sensor_write(unsigned char add, unsigned char val)
{
  P5OUT |= mSS;                         /* Deselect Select Memory as SPI slave */  
  P5OUT &= ~nSS;                        /* Select Sensor as SPI Slave */
  unsigned char RXCHAR = 0x00;
  
  RXCHAR = Sensor_TXRX(add, val);
  
  P5OUT |= nSS;                         /* Deselect Sensor as SPI slave */        
  return RXCHAR;
}

unsigned char _Sensor_read(unsigned char add)
{
  
  unsigned char RXCHAR = 0x00;
  P5OUT |= mSS;                         /* Deselect Select Memory as SPI slave */  
  P5OUT &= ~nSS;                        /* Select Sensor as SPI Slave */
  
  RXCHAR = Sensor_TXRX(add | MPU_READ, 0x00);
  
  P5OUT |= nSS;                         /* Deselect Sensor as SPI slave */   
  return RXCHAR;
}

unsigned char Sensor_TXRX(unsigned char add, unsigned char val)
{
  unsigned char RXCHAR = 0x00;
          
  while (!(UC1IFG & UCB1TXIFG));          /* Wait for TXBUF to be empty */
  
  UCB1TXBUF = add;                      /* Send Address of Register  */
  /* Wait for TXBUF to be empty (TXBUF data moves to the shift register) */
  while(!(UC1IFG & UCB1TXIFG));           
  while(!(UC1IFG & UCB1RXIFG));           /* Wait for RXBUF to be full */ 
  /* Wait for SPI to complete communication. Shouldn't be needed */
 // while(UCB1STAT & UCBUSY);             
  RXCHAR = UCB1RXBUF;                   /* Read what is RX to clear buffer / flags*/
  
  
  UCB1TXBUF = val;                      /* Write the val */      
  /* Wait for TXBUF to be empty (TXBUF data moves to the shift register) */
  while(!(UC1IFG & UCB1TXIFG));           
  while(!(UC1IFG & UCB1RXIFG));           /* Wait for RXBUF to be full */  
    
  RXCHAR = UCB1RXBUF;                     /* Read what is RX to clear buffer / flags*/
  
  return RXCHAR;
}


/**********************/
/* Code for Dataflash */
/**********************/
unsigned char MEM_TXRX(unsigned char data)
{
  unsigned char RXCHAR = 0x00;
  while (!(UC1IFG & UCB1TXIFG));          /* Wait for TXBUF to be empty */
  /* Wait for SPI to complete communication. Shouldn't be needed */
//  while(UCB1STAT & UCBUSY);             
  UCB1TXBUF = data;                      /* Send Address of Register  */
  /* Wait for TXBUF to be empty (TXBUF data moves to the shift register) */
  while(!(UC1IFG & UCB1TXIFG));           
  while(!(UC1IFG & UCB1RXIFG));           /* Wait for RXBUF to be full */ 
  /* Wait for SPI to complete communication. Shouldn't be needed */
//  while(UCB1STAT & UCBUSY);             
  RXCHAR = UCB1RXBUF;                   /* Read what is RX to clear buffer / flags*/
//  while(UCB1STAT & UCBUSY);
  return RXCHAR;
  
}

unsigned char Mem_ReadID()
{
  unsigned char RXCHAR = 0x00;
  P5OUT |= nSS;                         /* Deselect Sensor as SPI slave */ 
  P5OUT |= mSS;                         /* Deselect memory as SPI slave */ 
  P5OUT &= ~mSS;                        /* Select Memory as SPI Slave */
  
  MEM_TXRX(0x9F);                       /* Send the Buffer OpCode to the Memory */
  /* Write 3 bytes of address. We starts at byte 0, so this is always 0 */
  RXCHAR = MEM_TXRX(0x00);              
  RXCHAR = RXCHAR;
  
  P5OUT |= mSS;                         /* Deselect memory as SPI slave */ 
  return RXCHAR;
}

void Mem_WriteToBuffer()
{
  P5OUT |= nSS;                         /* Deselect Sensor as SPI slave */ 
  P5OUT |= mSS;                         /* Deselect memory as SPI slave */ 
  P5OUT &= ~mSS;                        /* Select Memory as SPI Slave */

  MEM_TXRX(Buf1Write);                  /* Send the Buffer OpCode to the Memory */
  /* Write 3 bytes of address. We starts at byte 0, so this is always 0 */
  MEM_TXRX(0x00);                       
  MEM_TXRX(0x00);
  MEM_TXRX(0x00);
  
  for(ctr = 0; ctr < PAGESIZE; ctr++)
    MEM_TXRX(SensorData[ctr]);

   /* Wait for SPI to complete communication. Shouldn't be needed */
  while(UCB1STAT & UCBUSY);            
  P5OUT |= mSS;                         /* Deselect memory as SPI slave */ 
  __delay_cycles(100);
}

void Mem_ReadFromBuffer()
{
  P5OUT |= nSS;                         /* Deselect Sensor as SPI slave */ 
  P5OUT |= mSS;                         /* Deselect memory as SPI slave */ 
  P5OUT &= ~mSS;                        /* Select Memory as SPI Slave */
  
  MEM_TXRX(Buf1Read);                  /* Send the Buffer OpCode to the Memory */
  MEM_TXRX(0x00);                      /* Don't Care Bytes */
   /* Upper and lower Byter. We starts at byte 0, so this is always 0 */
  MEM_TXRX(0x00);                     
  MEM_TXRX(0x00);                      /* Lower Byte */
  
  for(ctr = 0; ctr < PAGESIZE; ctr++)
      SensorData[ctr] = MEM_TXRX(0x00);
/* Wait for SPI to complete communication. Shouldn't be needed */
  while(UCB1STAT & UCBUSY);             
  P5OUT |= mSS;                         /* Deselect memory as SPI slave */ 
  
}

void Mem_BufferToPage()
{
  PageAddress_H = (unsigned char) (((CurrentPage<<2) & 0xFF00)>>8);
  PageAddress_L = (unsigned char) (((CurrentPage<<2) & 0xFF));
  
  P5OUT |= nSS;                         /* Deselect Sensor as SPI slave */ 
  P5OUT |= mSS;                         /* Deselect memory as SPI slave */ 
  P5OUT &= ~mSS;                        /* Select Memory as SPI Slave */
  
  MEM_TXRX(Buf1ToFlashWE);                  /* Send the Buffer OpCode to the Memory */
  MEM_TXRX(PageAddress_H);                      /* Don't Care Bytes */
  /* Upper and lower Byter. We starts at byte 0, so this is always 0 */
  MEM_TXRX(PageAddress_L);                      
  MEM_TXRX(0x00);                      /* Lower Byte */
  
  P5OUT |= mSS;                         /* Deselect memory as SPI slave */ 
}

void Mem_ReadFromMem(unsigned int PageToRead)
{
  PageAddress_H = (unsigned char) (((PageToRead<<2) & 0xFF00)>>8);
  PageAddress_L = (unsigned char) (((PageToRead<<2) & 0xFF));
  
  P5OUT |= nSS;                         /* Deselect Sensor as SPI slave */ 
  P5OUT |= mSS;                         /* Deselect memory as SPI slave */ 
  P5OUT &= ~mSS;                        /* Select Memory as SPI Slave */
  
  MEM_TXRX(0xD2);                  /* Send the Buffer OpCode to the Memory */
  MEM_TXRX(PageAddress_H);                      /* Don't Care Bytes */
  /* Upper and lower Byter. We starts at byte 0, so this is always 0 */
  MEM_TXRX(PageAddress_L);                      
  MEM_TXRX(0x00);                      /* Lower Byte and Add*/

  MEM_TXRX(0x00);                       /* Dummy Byte */
  MEM_TXRX(0x00);                       /* Dummy Byte */
  MEM_TXRX(0x00);                       /* Dummy Byte */
  MEM_TXRX(0x00);                       /* Dummy Byte */
  
  for(ctr = 0; ctr < PAGESIZE; ctr++)
      SensorData[ctr] = MEM_TXRX(0x00);
  /* Wait for SPI to complete communication. Shouldn't be needed */
  while(UCB1STAT & UCBUSY);             
  P5OUT |= mSS;                         /* Deselect memory as SPI slave */ 
}


/* Old function used when READMEM */
void Mem_ReadAllBinary()
{
    for(int i = 0; i < CurrentPage ; i++)
    {
      
      for(ctr = 0; ctr < PAGESIZE; ctr++)
          SensorData[ctr] = 0;
      
      Mem_ReadFromMem(i);
      
      for(ctr = 0; ctr < 170; ctr++)
        for(int ctr2 = 0; ctr2 < 6; ctr2++)
		UART_SendChar(SensorData[(ctr*6)+ctr2]);
    }
    
    while(UCA0STAT & UCBUSY);
}



void UART_SendChar(unsigned char data)
{ 
  while(!(IFG2 & UCA0TXIFG));
  UCA0TXBUF = data;  
}




/* Interrupt for TimerA Channel 0, runs short periods*/
#pragma vector = TIMERA0_VECTOR
__interrupt void TA0V_ISR(void)
{
  __disable_interrupt();
  TACCR0 += (BaseTime); /* Next Interrupt at given time */
  BaseTime ^= 0x01;             /* Flip between 2184 and 2185 to average at 2184.5 */
  
  ActionMode = 3;

  if(ReadingSensor == 1)
   JustDance(); 

  if( (SVSCTL & SVSFG) == SVSFG)
  {
    ActionMode = 1;
    FORCESTOP = 1;
  }
  else
  {
    FORCESTOP = 0;
  }
  __low_power_mode_off_on_exit();
  P1IFG = 0;                    /* Changing P1 can change P1IFG */
  __enable_interrupt();
}



/* Interrupt for Port 1 Button */
#pragma vector = PORT1_VECTOR
__interrupt void P1IV_ISR(void)
{
__disable_interrupt();  
  
  /* Temporarily disable the interrupt (This is enabled after timer compares) */
  P1IE &= ~BTN;                         
  switch(P1IFG)
  {
  case BTN:
          if(debounce == 1)
          {
              /* Set debounce to 0 to indicate that timer needs*/
              /* to overflow before button can be read again   */
              debounce = 0;                     
              TACCR1 = TAR + 0x1800;    /* Timer 1 for half second later */
              TACCTL1 = CCIE;           /* Enable Interrupt */        
              TACCR2 = TAR+0x9000;      /* Timer 2 for long press */
              TACCTL2 = CCIE;           /* Enable Timer 2 interrupt */
          }
          break;
          
  default: break;
  }
  
  P1IFG = 0;                    /* Clear any interrupts on P1 */
  P1IE  |= BTN;                 /* Set BTN as intterupt (default P1IES sets low to high )*/
  P1IES |= BTN;                 /* Sensitive to (H->L) */
  P1IFG = 0;                    /* Clear any interrupts on P1 */
  __enable_interrupt();
}

/* Interrupt for TimerA Channel 1 */
#pragma vector = TIMERA1_VECTOR
__interrupt void TAIV_ISR(void)
{
__disable_interrupt();    
switch(TAIV)
  {
  /* half a second has passed, so process the button command */
  case TAIV_TACCR1:                     
          TACCR1 = 0;
          TACCTL1 &= ~CCIE;
          /* half Seconds later, so we can say button has been debounced */
          debounce = 1;                 
          ActionMode = 2;               /* Action - Short Button Press */ 
          __low_power_mode_off_on_exit();
                 
          break;   
  case TAIV_TACCR2:
          /* 3 Seconds later after button press. */
          /* Turn Channel 1 to off. */
          
          TACCTL2 &= ~CCIE;
          TACCR2 = 0;
          ActionMode = 1;               /* Action = Long Button Press */

          debounce = 1; /* half Seconds later, so we can say button has been debounced */
          __low_power_mode_off_on_exit();   
            break;
  default:  break; 
  }
 
  P1IFG = 0;                    /* Clear any interrupts on P1 */
  P1IE  |= BTN;                 /* Set BTN as intterupt (default P1IES sets low to high )*/
  P1IES |= BTN;                 /* Sensitive to (H->L) */
  P1IFG = 0;                    /* Clear any interrupts on P1 */
}

void JustDance()
{
 /* Turn LED off if the WRONG WHO_AM_I response is received */
 P1OUT &= ~LED;                   
      for(int j = 0; j<0x05; j++) for(int i = 0; i<0xFFFF; i++);        /* Long LED off delay */
      P1OUT ^= LED;
      for(int j = 0; j<0x05; j++) for(int i = 0; i<0xFFFF; i++);        /* Long LED on */
      P1OUT ^= LED;
      for(int i = 0; i<0x8FFF; i++);        /* Short LED off */
      P1OUT ^= LED;
      for(int i = 0; i<0x8FFF; i++);        /* Short LED on */
      P1OUT ^= LED;
      for(int i = 0; i<0x8FFF; i++);        /* Short LED off */
      P1OUT ^= LED;
      for(int i = 0; i<0x8FFF; i++);        /* Short LED on */
      P1OUT ^= LED;
      for(int i = 0; i<0x8FFF; i++);        /* Short LED off */
      P1OUT ^= LED;
      for(int i = 0; i<0x8FFF; i++);        /* Short LED on */
      P1OUT ^= LED;
      for(int j = 0; j<0x05; j++) for(int i = 0; i<0xFFFF; i++);        /* Long LED on delay */
//      P1OUT ^= LED;                                                     
  P1IFG = 0;                    /* Clear any interrupts on P1 */
  P1IE  |= BTN;                 /* Set BTN as intterupt (default P1IES sets low to high )*/
  P1IES |= BTN;                 /* Sensitive to (H->L) */
  P1IFG = 0;                    /* Clear any interrupts on P1 */
}

// USCI A0/B0 Receive ISR
#pragma vector=USCIAB0RX_VECTOR
__interrupt void USCI0RX_ISR(void)
{
__disable_interrupt();
ReadPageNumberFromFlash();
//UC1IE &= ~UCA1RXIE;                          // Disable RX interrupt
UART_data[0] = UART_data[1];
UART_data[1] = UCA0RXBUF;

if(UART_data[0] == 's' && UART_data[1] == 'd')
{
  UART_data[0] = 0x00;
  UART_data[1] = 0x00;
  Mem_ReadAllBinary();
}
__enable_interrupt();
//UC1IE |= UCA1RXIE;                          // Enable  RX interrupt
}

void ReadPageNumberFromFlash()
{
  char *Flash_ptr;                          // Flash pointer
  unsigned char LSB,MSB;
  
  Flash_ptr = (char *)0x1040;              // Initialize Flash segment C ptr
  MSB = *Flash_ptr++;                // Get Current Page number from the Flash location 0x1040
  LSB = *Flash_ptr;
  CurrentPage = (((unsigned int)MSB)<<8)+(unsigned int)LSB;
  __no_operation();                       // SET BREAKPOINT HERE
}

void WritePageNumberToFlash()
{
  char *Flash_ptr;                          // Flash pointer
  unsigned char LSB,MSB;
  Flash_ptr = (char *)0x1040;              // Initialize Flash segment C ptr
  
  FCTL3 = FWKEY;                            // Clear Lock bit
  FCTL1 = FWKEY + ERASE;                    // Set Erase bit
  *Flash_ptr = 0;                          // Dummy write to erase Flash seg D
  FCTL1 = FWKEY + WRT;                      // Set WRT bit for write operation
  
  MSB = (unsigned char)((CurrentPage & 0xFF00u)>>8);
  LSB = (unsigned char)(CurrentPage & 0xFF);
  *Flash_ptr++ = MSB;
  *Flash_ptr = LSB;
  
  __no_operation();                       // SET BREAKPOINT HERE
  
  FCTL1 = FWKEY;                            // Clear WRT bit
  FCTL3 = FWKEY + LOCK;                     // Set LOCK bit
}


void OFFDANCE()
{
        P1OUT &= ~LED;
        for(int j = 0; j<0x05; j++) for(int i = 0; i<0xFFFF; i++);        /* Long LED off delay */
        P1OUT ^= LED;
        for(int j = 0; j<0x05; j++) for(int i = 0; i<0xFFFF; i++);        /* Long LED on */
        P1OUT ^= LED;
        for(int j = 0; j<0x05; j++) for(int i = 0; i<0xFFFF; i++);        /* Long LED off */
        P1OUT ^= LED;
        for(int j = 0; j<0x05; j++) for(int i = 0; i<0xFFFF; i++);        /* Long LED on */
        P1OUT ^= LED;
        for(int j = 0; j<0x05; j++) for(int i = 0; i<0xFFFF; i++);        /* Long LED off */  
}


void ONDANCE()
{
        P1OUT &= ~LED;
        for(int i = 0; i<0x8FFF; i++);        /* Long LED off delay */
        P1OUT ^= LED;
        for(int i = 0; i<0x8FFF; i++);        /* Long LED on */
        P1OUT ^= LED;
        for(int i = 0; i<0x8FFF; i++);        /* Long LED off */
        P1OUT ^= LED;
        for(int i = 0; i<0x8FFF; i++);        /* Long LED on */
        P1OUT ^= LED;
        for(int i = 0; i<0x8FFF; i++);        /* Long LED off */  
}
\end{lstlisting}