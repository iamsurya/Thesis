This section covers the actual methods and techniques used to create our wrist 
motion activity tracker. We discuss the various hardware techniques used to 
create the device in section \ref{Sec:Hardware}.
This section covers researching on the various sensors available in the market,
picking the right one for our purpose and then prototyping a device.
Since the wrist activity motion tracker is to be mounted on the wrist, it needs
to be a small device that can be worn for a long interval\footnote{At least
16 hours} without discomfort to the user.% This means that it also needs to
%have a long battery life, which is dictated by two factors: The size of the 
%battery, and the current consumption of the parts on the device.

Next in this section, we discuss the software techniques and code
used to create the device or the software to interact with it that runs on a desktop computer.
Two different software were used to analyze the data from the device.
This is briefly discussed in the Software section (section \ref{Sec:Software}).
Dong~\cite{dong2013detecting}, showed that rotational and linear movement data can be used to detect
periods of eating. This made us look towards the rotational and linear movements of the wrist.
To capture this information, we consider devices like Accelerometers and Gyroscopes, which can measure such movements.

\section{Hardware Selection}
\label{Sec:Hardware}